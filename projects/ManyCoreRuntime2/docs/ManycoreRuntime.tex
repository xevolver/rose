\documentclass[10pt]{article}
%\documentclass[10pt]{book}

\usepackage[draft]{fixme}

\usepackage{listings}
\usepackage{html}
\usepackage{color}
\usepackage{multicol}
\usepackage{multirow}
\usepackage{graphicx}
\usepackage{alltt}
% style for code listing
\lstset{language={C},basicstyle=\scriptsize} 
\newcommand{\hlstd}[1]{\textcolor[rgb]{0,0,0}{#1}}
\newcommand{\hlkey}[1]{\textcolor[rgb]{0,0,0}{\bf{#1}}}
\newcommand{\hlnum}[1]{\textcolor[rgb]{0.16,0.16,1}{#1}}
\newcommand{\hltyp}[1]{\textcolor[rgb]{0.51,0,0}{#1}}
\newcommand{\hlesc}[1]{\textcolor[rgb]{1,0,1}{#1}}
\newcommand{\hlstr}[1]{\textcolor[rgb]{1,0,0}{#1}}
\newcommand{\hldstr}[1]{\textcolor[rgb]{0.51,0.51,0}{#1}}
\newcommand{\hlcom}[1]{\textcolor[rgb]{0.51,0.51,0.51}{\it{#1}}}
\newcommand{\hldir}[1]{\textcolor[rgb]{0,0.51,0}{#1}}
\newcommand{\hlsym}[1]{\textcolor[rgb]{0,0,0}{#1}}
\newcommand{\hlline}[1]{\textcolor[rgb]{0.33,0.33,0.33}{#1}}

\newcommand{\mySmallFontSize}{\scriptsize}
\newcommand{\mySmallestFontSize}{\tiny}

\newcommand{\codeFontSize}{\scriptsize}
\newcommand{\code}[1]{{\scriptsize #1}}

% Software version number
%\newcommand{\VersionNumber}{@VERSION@}

%\newcommand{\ExampleDirectory}{@top_srcdir@/projects/compass/tests}

% Latex trick to allow us to comment out large sections of documentation
\newcommand{\commentout}[1]{}

% change the title of the Fixme List
\renewcommand{\listfixmename}{Things to Fix in Documentation}

\newcommand{\comm}[2]{\bigskip
                      \begin{tabular}{|p{11cm}|}\hline
                      \multicolumn{1}{|c|}{{\bf Comment by #1}}\\ \hline
                      #2\\ \hline
                      \end{tabular}
                      \bigskip
                     }

\def\verbatimfile#1{\begingroup
                    \@verbatim \frenchspacing \@vobeyspaces
                    \input#1 \endgroup
}

\newcounter{lineno}

% Taken from verbatimfiles.sty on web
\makeatletter %JCL

\def\verbatimlisting#1{\setcounter{lineno}{0}%
    \begingroup \@verbatim \frenchspacing \@vobeyspaces \parindent=20pt
    \everypar{\stepcounter{lineno}\llap{\thelineno\ \ }}\input#1
    \endgroup
}

\makeatother %JCL

% \endinput

\addtolength{\textheight}{0.5in}
\sloppy

%---------------------------------------------------------------------
% Begin Document
%---------------------------------------------------------------------

\begin{document}

% This draft mode eliminates the figures (leaves boxes for where they would be)
%\textcolor{green}{(Associated with ROSE Version @VERSION@)} } }
%\psdraft

\title{ {\bf \textcolor{red}{A Compiler-Independent Many-Core Runtime System} \\ 
              \textcolor{blue}{Draft User Tutorial} \\
              } }
\author{ {\bf ROSE Team} \\
         Lawrence Livermore National Laboratory \\ 
         Livermore, CA  94550 \\
         925-423-2668 (office)  925-422-6278 (fax) \\
         \{liao6, quinlan1\}@llnl.gov \\
         Project Web Page:
         \htmladdnormallink{http://www.rosecompiler.org}{http://www.rosecompiler.org} \\
         UCRL Number for ROSE User Manual: UCRL-SM-210137-DRAFT \\
         UCRL Number for ROSE Tutorial: UCRL-SM-210032-DRAFT \\
         UCRL Number for ROSE Source Code: UCRL-CODE-155962 \\ \\
         \htmladdnormallink{ROSE User Manual
         (pdf)}{http://www.rosecompiler.org/ROSE_UserManual/ROSE-UserManual.pdf} \\
         \htmladdnormallink{ROSE Tutorial
         (pdf)}{http://www.rosecompiler.org/ROSE_Tutorial/ROSE-Tutorial.pdf} \\
         \htmladdnormallink{ROSE HTML Reference (html
         only)}{http://www.rosecompiler.org/ROSE_HTML_Reference/index.html}
       }
\maketitle
\newpage


% This fixes the really long table of contents problem
\setcounter{tocdepth}{2}

\tableofcontents
\newpage
%

%\chapter{Introduction}
\section{Introduction}
This document describes an on-going project aimed at defining a compiler independent
runtime system for many-core optimization.  It is presently distributed as 
part of the ROSE compiler~\cite{roseWeb2008}, but nothing about it is dependent 
upon the ROSE compiler. It is also part of a few projects that are specific to 
Exascale research.

This document is not meant to reflect the final design or implementation choices. 


This project represents a runtime system to support many core optimizations.

A specific focus is on exascale architectues and supporting manycore
processors that will have either no cross-chip cache coherency or 
poor performance of any possible cross-chip cache coherency.

As a model exascale machines we assume:
\begin{enumerate}
   \item A million processors (this runtime system is mostly independent of distributed
      memory processing).

   \item A thousand cores per processor (to be thought of as any combination of CPU cores
    or GPU cores).

   \item Zero cross-chip cache coherency.
\end{enumerate}


This work is a draft of an evlving many-core runtime system that will fit with 
other programming model building blocks.


A goal if for this work to be coupled with a compiler to support many-core
optimizations for HPC applications.  Stencil operations are a specific target.


\vspace{1.0}
Related papers:
\begin{enumerate}
   \item Compiler Techniques for the Distribution of Data and Computation
      Naverro, Zapata, and David Padua.
   \item The Galois project at University of Texas at Austin supports irregular 
      computations and is relevant for how we might handle irregular computations
      within the runtime system.  It is not clear yet if we can use it directly 
      or just indirectly, but it is clearly relavant work. Since it is now implemented
      in C++, it is likely more campatable with our use then when it was previously
      implemented in Java (more info at: http://iss.ices.utexas.edu/?p=projects/galois).

\end{enumerate}




\bibliographystyle{plain}
\bibliography{minidb}
\listoffixmes

\end{document}

%
